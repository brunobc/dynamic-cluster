Arboviroses são doenças causadas pelos chamados arbovírus, ou seja, vírus que vivem essencialmente em artrópodes como o Aedes aegypti, os causadores de doenças como a dengue, Zika, febre chikungunya e febre amarela. Essas doenças estão cada vez mais perigosas à população redobrando a atenção de orgãos internacionais \acrshort{OMS}-\acrshort{TDR} e das autoridades de saúde brasileiras para um esforço maior na prevenção e combate a sua expansão. Esta pesquisa fez importantes esforços para desenvolver, projetar e implementar uma estrutura computacional baseada na web, que visa apoiar, ajudar a rastrear e gerenciar os recursos e pessoas no processo de prevenção e combate à arboviroses. Além disto, apresenta uma abordagem para mitigar o problema da previsão de agrupamentos dinâmicos espaço-temporal delineado pelos casos humanos, e combater as arboviroses com o esforço coordenado de uma estrutura de Sistemas de Apoio à Decisão que rastrea simultaneamente o mosquito e os casos em humanos nos territórios afetados.
Como se trata de um sistema de difícil previsão espaço-temporal de sua ocorrência, foram implementados dois métodos para visualização dos grupos formados de casos humanos.
O primeiro é uma biblioteca que cria e gerencia grupos de acordo com o nível de proximidade.
O segundo, consiste em gerar a região no espaço-tempo que melhor representa um grupo de casos de similaridade. Algoritmos de agrupamento espaço temporais ST-DBSCAN e ST-IGN foram implementados como base para consolidar a metodologia proposta.
Por fim, foram desenvolvidos  aprimoramentos no software de visualização espaço temporal, Dynagraph para possibilitar a avaliação dos métodos e seu uso corrente. Um deles foi a integração com um Editor de
Características Espaço-Temporal, que permite alterar os atributos visuais dos vértices e arestas de um
grafo dinâmico. Outro recurso permite a visualização da formação de novos grupos dinâmicos baseado no histórico dos grupos.
Como resultado, os métodos chegaram a apresentar níveis de assertividade acima de 98\% nas situações mais críticas avaliadas da dengue e chikungunya no município de Fortaleza/CE.


% Separe as palavras-chave por ponto
\palavraschave{Agrupamento Dinâmicos, Grafos dinâmicos, Modelo de Previsão}