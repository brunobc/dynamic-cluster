Arboviruses are diseases caused by so-called arboviruses, that is, viruses that live primarily on arthropods such as Aedes aegypti, the cause of diseases such as dengue, Zika, chikungunya fever and yellow fever. These diseases are increasingly dangerous to the population, redoubling the attention of international bodies WHO-TDR and Brazilian health authorities to a greater effort to prevent and combat its expansion. This research has made important efforts to develop, design and implement a web-based computational structure that aims to support, help track and manage resources and people in the process of arbovirus prevention and control.
In addition, it presents an approach to solve the problem of predicting space-time dynamic clusters and to combat arboviruses with the coordinated effort of a Decision Support Systems structure to simultaneously trace the mosquito and cases in humans to prevent and combat affected territories.
As it is a system of difficult space-time prediction of its occurrence, two methods were implemented to visualize the formed groups of human cases.
The first is a library that creates and manages groups according to the level of proximity.
The second is to generate the region in space-time that best represents a group of similarity cases.
ST-DBSCAN and ST-IGN temporal space grouping algorithms were implemented as a basis to consolidate the proposed methodology. Finally, improvements were developed in the visualization software, Dynagraph, to enable the evaluation of the methods and their current use. One of them was the integration with a Space-Time Character Editor, which allows changing the visual attributes of the vertices and edges of a dynamic graph. Another feature allows visualization of the formation of new dynamic groups based on the history of the groups.
As a result, the methods showed levels of over 98\% in the most critical situations of dengue and chikungunya in the city of Fortaleza/CE.

\keywords{Dynamic Clustering, Dynamic Graphs, Prediction Model}