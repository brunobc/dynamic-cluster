\section{Estudos Relacionados e Escolha de Métodos Dinâmicos}
\label{sec:trabalhos-relacionados}

A potência de qualquer tipo de abordagem de rede reside na capacidade de simplificar um sistema complexo para que se possa compreender melhor a sua função como um todo. Às vezes, é vantajoso, no entanto, incluir mais informação em um grafo simples do que ter apenas nós e ligações. Adicionando informações sobre tempos de interações podem fazer previsões e entendimentos mais precisos. A desvantagem, contudo, é que não há tantos métodos disponíveis, em parte porque as redes temporais é um campo relativamente novo, em parte porque é mais difícil de desenvolver tais métodos em comparação com as redes estáticas \cite{holme:colloquium}.

\subsection{Estudos Relacionados}
\label{subsec:estudos}

\citeautoronline{DanielNeill2006} desenvolveu um \emph{framework} com novos métodos estatísticos e computacionais para a detecção automática de \emph{clusters} espaciais e espaço-temporais. Esse \emph{framework} de “varredura espacial generalizada” é uma estrutura flexível baseada em um modelo para detecção de \emph{cluster} de forma precisa e computacionalmente eficiente em diversos domínios de aplicação, através do desenvolvimento do algoritmo de “varredura espacial rápida” e novos métodos Bayesianos de detecção de \emph{clusters}. Esses métodos de detecção de \emph{cluster} foram aplicados à detecção de epidemias de doenças emergentes. Por exemplo, no domínio da saúde pública, pode-se querer detectar agrupamentos espaciais de casos de doença (ou alguma quantidade observável relacionada, como visitas a hospitais ou vendas de medicamentos) que são indicativos de uma epidemia emergente.

\citeautoronline{Mitsa:2010} abrange a teoria da mineração de dados temporais, bem como aplicações em uma variedade de áreas do conhecimento e campos de atuação nestas áreas. Seus objetivos foram: fornecer os conceitos básicos, bem como o estado da arte, e discutir as aplicações e avanços do estado da arte da mineração de dados temporais. Os conceitos básicos estudados por eles foram:
\begin{itemize}
\item Incorporação de temporalidade em bancos de dados.
\item Representação de dados temporais.
\item Classificação de dados temporais e agrupamento.
\item Descoberta de padrão temporal.
\item Predição.
\end{itemize}
Não houve aqui a caracterização de um \emph{framework} espaço-temporal para qualquer aplicação, específica que norteasse a elaboração de um trabalho neste sentido. 

\citeautoronline{Taiwan2010} analisam se os padrões espaço-temporais da dengue podem ser usados para identificar áreas de risco de dengue hemorrágica.
A pesquisa utiliza três índices como métodos: probabilidade de ocorrência de casos, duração média por onda de epidemia e intensidade de transmissão. Foram usados oito locais de dengue para estudar os padrões espaço-temporais durante a epidemia de 2002 em Kaohsiung, Taiwan, onde foram comparadas as densidades espaço-temporais da dengue hemorrágica em cada área. Os três índices espaço-temporais de dengue podem fornecer informações úteis para identificar áreas com alto risco de dengue hemorrágica.

\citeautoronline{Shekhar2011} revisaram técnicas e ferramentas computacionais recentes na mineração de dados espaço-temporais, focando em vários tipos de padrões: \textit{outliers}, acoplamento, previsão, particionamento e sumarização, \textit{hotspot} e detecção de mudanças. Em comparação com outras pesquisas da literatura, esse artigo enfatiza os fundamentos estatísticos da mineração de dados espaço-temporais e fornece uma cobertura abrangente de abordagens computacionais para vários tipos de padrões. Também listam ferramentas de \textit{software} populares para análise de dados espaço-temporais, por exemplo, o \textit{software} ArcGIS \cite{ArcGIS}, onde \acrshort{GIS} representa um sistema de informação geográfica, utilizado para trabalhar com mapas e que possui uma extensão chamada \emph{Tracking Analyst} para suportar visualização e análise de dados espaço-temporais; a ferramenta estatística espacial \emph{R} \cite{R}, que fornece pacotes para análise estatística espacial e espaço-temporal; o CrimeStat \cite{Levine2017CrimeStat}, um pacote de \textit{software} para análise espacial de locais de crime e que incorpora vários métodos de agrupamento para determinar os \emph{hotspots} de crimes em uma área de estudo.


\citeautoronline{ping2012} aborda três aspectos dos estudos de saúde: detecção de \textit{clusters} de doenças espaciais, mapeamento da doença espaço-temporal e planejamento de serviços de saúde. A pesquisa propõe o uso do algoritmo \acrshort{RSScan} (\emph{\acrlong{RSScan}}) para detectar \textit{clusters} de doenças em formas arbitrárias. Para explorar os padrões espaço-temporais dos riscos de incidência de câncer de pulmão no estado da Geórgia/EUA entre 2000 e 2007. Sete modelos bayesianos hierárquicos são desenvolvidos e comparados ao nível de setor censitário usando um período de dois anos como unidade temporal. Dois modelos de transporte que abordam o problema do local de cobertura máximo capacitado modular (\acrshort{MCMCLP}) são propostos e usados para posicionamento ótimo de ambulâncias para a Região 10 dos \acrfull{EMS} na Geórgia/EUA.

\citeautoronline{eric2014} aplicam uma extensão espacial e temporal do algoritmo \acrfull{EDK} para mapear \textit{clusters} espaço-temporais de dengue em Cali, Colômbia, no primeiro semestre de 2010. Usam técnicas de agrupamento baseados em \emph{Scan Statistics} \cite{Kulldorff1997}, a partir do modelo espaço-temporal estimador de núcleos de densidade (\acrshort{STKDE}), uma extensão do modelo de \citeautoronline{Silverman86}. A Estatística Scan (\emph{Scan Statistics}) é comumente usada para identificar possíveis \emph{clusters} cujos centros de massa (centróide) estejam dentro de um polígono com raio da região variando conforme o percentual da população dentro do grupo. Para determinar o impacto nos \textit{clusters} espaço-temporais causado por erros de geocodificação assim como erros no tempo estimado da infecção, a localização de cada caso e o tempo relatado de infecção são analisados usando dados empíricos para controlar a magnitude do distúrbio da doença. Essa pesquisa usa uma abordagem de computação paralela, que permite que o problema seja resolvido dentro de um prazo mais gerenciável. Os dados gerados são visualizados usando um ambiente 3D, revelando períodos e áreas de ocorrências de alta densidade de dengue. Os volumes de densidade mínima e máxima são gerados para cada um dos \textit{clusters} predominantes nas áreas de densidade de ocorrência.


\citeautoronline{Zhicheng:2019} focam nos métodos de agrupamento onde os eventos possuem os tipos de dados espaço-temporal, analisam que esses dados podem ser classificados em três categorias: ponto, linha e polígono, e os métodos de agrupamento existentes são baseados em testes de hipóteses e métodos de agrupamento particionais. Os dados espaço-temporal são mais complicados do que outros tipos de dados devido à dimensão adicional de tempo para a análise espacial bidimensional.


\subsection{Escolha dos Métodos Dinâmicos}
\label{subsec:metodos-escolhidos}

Como vimos na seção anterior há uma carência de estudos relacionando os assuntos abordados: agrupamento, previsão em dados dinâmicos espaço-temporais, grafos dinâmicos e sistemas web de forma integrada. Deste modo foi necessário dividir o problema de agrupamentos e previsões dinâmicos em três etapas:
\begin{itemize}
\item Estrutura de dados em grafos dinâmicos.
\item Modelos de previsão espaço-temporais.
\item Algoritmos de agrupamento dinâmico.
\end{itemize}

Esta pesquisa aborda nas seções anteriores deste capítulo a estrutura de dados em grafos dinâmicos usando passos já descritos na literatura, principalmente o modelo Dynagraph \cite{dynagraph}, que é baseado na primeira proposta em \cite{dynagraph2012}, onde o Dynagraph usa sequências temporais para vértices, arestas, características modificáveis dos vértices e arestas e o relacionamento entre suas características. Ele permite formar um grafo com as informações necessárias para qualquer instante no tempo. 

O algoritmo de agrupamento dinâmico \acrshort{ST-DBSCAN} tem como base o algoritmo \acrshort{DBSCAN}, o qual possui a capacidade de descobrir agrupamentos de acordo com valores não espaciais, espaciais e temporais dos objetos. Com ele é possível descobrir grupos em dados espaciais-temporais baseados na densidade das ocorrências dos objetos. Tais agrupamentos se caracterizam por se distribuem em massas compactas, pois são baseadas em densidade. Logo, este método pode ser usado no contexto da descoberta de grupos para o conjunto de dados espaço-temporais de casos humanos de arboviroses, dada a forma como elas se expandem. 

O método \acrshort{ST-IGN} consiste da aplicação do método \acrshort{IGN} estático \cite{simposioNeg2003} sobre cada uma das $n$ instâncias (estados) geradas a partir de um período (intervalo fixo de estados) definido pelo processo de identificação da dinâmica do comportamento dos objetos. O método rotula os grupos naturais encontrados na unidade de tempo $i$ com base na máxima densidade de objetos que se mantém no mesmo grupo após uma divisão e/ou aglomeração.

Os dois algoritmos apresentados são utilizados como base para os modelos de previsão espaço-temporais. Um modelo simples de previsão é implementado, o qual se basea no histórico dos grupos, a partir do centro e do raio dos grupos em análise, onde o centro e o raio são definidos de acordo com a média da posição dos objetos de cada grupo identificado no tempo. Dos grupos identificados como sendo de mesma origem no tempo são então tomados os valores médios de seus centros e raios, gerando finalmente o grupo de previsão do próximo intervalo. O método \acrshort{ST-IGN} em especial tem a vantagem da separabilidade dos grupos naturais, caso exista de forma bem definida, como foi mostrado no capítulo \ref{chap:estadodaarte}, além disto este método consegue descobrir grupos que se formam em linha e mesmo em massa, apesar da desvantagem de formar grandes grupos quando as distãncias são muito pequenas entre objetos.

Uma vez que é bastante assertivo nos grupos que forma, não importando a quantidade de dados, trata-se de um método que pode ser considerado para avaliação, haja vista sua liberdade em encontrar grupos de várias formas (circulares, longitudinais e outras) permitindo-nos abrir a caracterização de padrões de comportamento das arboviroses, além das hoje consideradas pela literatura.