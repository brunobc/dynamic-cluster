\section{Trabalhos Relacionados}
\label{sec:trabalhos-relacionados}

% TODO: melhorar este capítulo
Como há uma carência de estudos relacionando os assuntos abordados: agrupamento, previsão em dados dinâmicos espaço-temporais, grafos dinâmicos e sistemas web de forma integrada, foi necessário dividir o problema de agrupamentos e previsões dinâmicos em três etapas:
\begin{itemize}
\item Estrutura de dados em grafos dinâmicos.
\item Modelos de previsão espaço-temporais.
\item Algoritmos de agrupamento dinâmico.
\end{itemize}

A pesquisa aborda a estrutura de dados em grafos dinâmicos usando passos já descritos na literatura, principalmente o modelo Dynagraph \cite{dynagraph}, que é baseado na primeira proposta em \cite{dynagraph2012}, onde o Dynagraph usa sequências temporais para vértices, arestas, características modificáveis dos vértices e arestas e o relacionamento entre suas características. Ele permite formar um grafo com as informações necessárias para qualquer instante no tempo. Com o Dynagraph é possível visualizar o comportamento do grafo ao longo de um período de tempo, verificar o estado de cada objeto a qualquer tempo e editá-lo.

O algoritmo de agrupamento dinâmico ST-DBSCAN tem como base o algoritmo DBSCAN, e tem a capacidade de descobrir
agrupamentos de acordo com valores não espaciais, espaciais e temporais dos objetos. Com ele é possível descobrir grupos em dados espaciais-temporais.

O método IGN dinâmico consiste da aplicação do método IGN estático \cite{simposioNeg2003} sobre cada uma das $n$ instâncias (estados) geradas a partir de um período (intervalo fixo de estados) definido pelo processo de identificação da dinâmica do comportamento dos objetos. O método rotula os grupos naturais encontrados na unidade de tempo $i$ com base na máxima densidade de objetos que se mantém no mesmo grupo após uma divisão e/ou aglomeração.

Os dois algoritmos apresentados são utilizado como base para os modelos de previsão espaço-temporais. Um modelo simples de previsão é implementado, e que é baseado no histórico dos grupos, a partir do centro e do raio dos grupos em análise, onde o raio é definido de acordo com a média do número de casos de dengue ou Chikungunya de cada grupo.