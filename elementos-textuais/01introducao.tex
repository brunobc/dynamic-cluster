\chapter{Introdução}
\label{chap:introducao}
Grandes quantidades de dados estão disponíveis para análise em organizações hoje em dia. Estas enfrentam vários desafios quando se tenta analisar dados gerados com o objetivo de extrair informações úteis.

Esta capacidade analítica precisa ser reforçada com ferramentas capazes de lidar com eles sem tornar o processo de análise uma tarefa árdua.
Agrupamento de dados normalmente são usados no processo de análise de dados, pois esta técnica não exige qualquer conhecimento prévio. Contudo, os algoritmos de agrupamento geralmente requerem um ou mais parâmetros de entrada que influenciam o processo de agrupamento e os resultados que podem ser obtidos. 

Nos últimos anos, o problema de agrupamento dinâmico tem atraído o interesse de pesquisas, impulsionado pelo aumento da disponibilidade de grandes conjuntos de dados contendo elementos espaciais e temporais. Este problema pode ser analisado como um problema de otimização. Seu objetivo principal é maximizar as diferenças das características dos indivíduos de grupos distintos, e minimizar as diferenças das características dos indivíduos de um mesmo grupo.

Agrupamento de dados ganhou uso muito difundido, especialmente para dados estáticos. No entanto, o rápido crescimento de dados espaço-temporais de inúmeros instrumentos, como os satélites em órbita terrestre, criou uma necessidade de métodos de agrupamento espaço-temporais para extrair e monitorar agrupamentos dinâmicos. O agrupamento espaço-temporal dinâmico enfrenta dois grandes desafios: primeiro, os \textit{clusters} são dinâmicos e podem mudar de tamanho, forma e propriedades estatísticas ao longo do tempo. Em segundo lugar, vários dados espaço-temporais são incompletos, ruidosos, heterogêneos e altamente variáveis sobre espaço e tempo.


O problema de agrupamento dinâmico é o problema de encontrar grupos de dados de máxima dissimilaridade que evoluem no tempo e no maior tempo possível (ou com a maior estabilidade possível).

Já o problema de previsão de grupos dinâmicos introduz o conceito de indicar os possíveis grupos que serão formados no tempo após um conjunto de eventos serem observados previamente.

Métodos de previsão de agrupamentos dinâmicos têm sido estudados nas áreas de climatologia, detecção de doenças, entre outras. Estes estudos concentram-se em modelos espaço temporais que indicam caminhos de solução baseados na densidade espaço temporal dos dados. Nesta dissertação pretende-se estudar o comportamento de um método desta natureza, considerando uma aplicação no monitoramento do avanço dos casos humanos de doenças provocadas por arboviroses (dengue e chickungunya) em Fortaleza/CE.

A representação dinâmica dos eventos também tem sido um problema importante. Muitos modelos são utilizados tais como representação espaço temporal por vídeos e grafos. Na representação espaço temporal por vídeos, o objetivo é sincronizar os modelos de previsão com a descoberta dos agrupamentos enquanto um vídeo real acontece (usado principalmente na climatologia  \cite{faghmous2013}); já a representação dinâmica por grafos dinâmicos, tem sido estudada pela literatura com maior profundidade pois permite acompanhar eventos e gerar previsões espaço temporais sabendo-se e controlando-se os principais atores do processo, tais como: elemento dinâmico e contexto dinâmico como descritos em \cite{holme:predictability} e \cite{Mitsa:2010}.

Ferramentas de grafos dinâmicos foram desenvolvidas e seguem em uso, como o Gephi e o Dynagraph \cite{dynagraph}. Aqui será feito um trabalho sobre o modelo Dynagraph para representação de uma estrutura dinâmica incluindo a possibilidade de variação dinâmica das características (cor, forma, tamanho, etc) dos elementos do grafo. Deste modo, esta visão evolutiva ajudaria ao modelo sistêmico representar claramente a realidade da evolução dos fenômenos que se deseja demonstrar para facilitar a tomada de decisão.

\section{Objetivos}
\label{sec:objetivos}
A seguir, são expostos os objetivos desta dissertação, definindo o produto
final a ser obtido.

\subsection{Objetivo Geral}

Propor um modelo de grafos dinâmicos que inclua a possibilidade de descrever características dinâmicas de vértices e arestas, e de um modelo de previsão de agrupamentos dinâmicos para o monitoramento da evolução de doenças provocadas por arboviroses (Dengue e Chikungunya).

\subsection{Objetivos Específicos}
\label{sec:objetivos-especificos}

Para que se alcance o objetivo geral, as seguintes metas foram estabelecidas:

\begin{alineas}
	\item Extração de características de previsão espaço-temporal sobre a evolução dos agrupamentos dinâmicos.
	\item Avaliação dos resultados sobre bases reais ligadas a evolução de casos de Dengue e Chikungunya na cidade de Fortaleza/CE e outras bases dinâmicas.
\end{alineas}

\section{Hipóteses}
As hipóteses a seguir conduziram a elaboração desta dissertação:
\begin{alineas}
    \item É viabilizado o armazenamento dos dados por uma ferramenta de extração e armazenamento de dados automatizada seguindo a estrutura de dados do Dynagraph;
	\item É exequível a integração de um editor de características ao Dynagraph, que é um software extensível.	
	\item É realizável a utilização do modelo proposto de agrupamentos em grafos dinâmicos em um ambiente Web.	
	\item É possível a criação de um algoritmo capaz de sugerir  agrupamentos naturais de casos de doenças geolocalizados baseados no tempo.
\end{alineas}

\section{Justificativa}
Percebe-se a necessidade de ferramentas e estudos 
relacionando os assuntos abordados: agrupamento, previsão em dados
dinâmicos espaço-temporais, grafos dinâmicos e sistemas web de forma integrada.
E também, acelerar técnicas de agrupamento em grafos dinâmicos para tomada de decisão.

Em \cite{simda}, estão disponíveis dados espaço-temporais para acompanhar a evolução da dengue e chikungunya em Fortaleza, que é um sistema para monitorar diariamente os casos de endemias como zika, leishmaniose, leptospirose, dengue e chikungunya. Somente estes dois últimos estão disponíveis visualmente num mapa. As informações nesse sistema pouco ajudam na melhor tomada de decisão para direcionamento dos recursos apropriados para o controle das doenças nos períodos de grande crescimento. A antevisão do processo mostra a possibilidade de tornar factível a tomada de decisão assertiva. É preciso também saber o quão preciso é o método, para que seja usado como ferramenta de decisão neste contexto.

A relevância da pesquisa está em permitir uma análise dos dados extraídos 
para apoio à tomada de decisão,  concentrando-se na avaliação dos resultados sobre bases de dados
dinâmicas relativas a casos de Dengue e Chikungunya.
A pesquisa toma como base as características de evolução dos casos da doença
observados entre 2015 e 2018 em Fortaleza/CE.
Os dados foram tomados a partir do Sistema de Monitoramento Diário de Agravos(SIMDA), onde um estado é definido como o período de uma semana. 

\section{Metodologia}

As fontes principais de pesquisa foram \textit{sites} especializados em pesquisas científicas, por exemplo, o portal de periódicos
da CAPES, IEEE e outros sites de referências que possuem livros, periódicos e dissertações disponíveis.
Os temas essenciais abordados na pesquisa foram:
\begin{itemize}
	\item Estrutura de dados em grafos dinâmicos
	\item Algoritmos de agrupamentos dinâmicos
	\item Modelos de previsão espaço-temporais
\end{itemize}

No trabalho proposto, buscou-se inicialmente extrair informações baseadas na previsão de agrupamentos dinâmicos
em grafos. Assim sendo, e após obter os dados a partir do \cite{simda}, através de uma ferramenta implementada que automatiza a extração e armazenamento dos dados, detectou-se a necessidade de um software para representação e tratamento de grafos dinâmicos, que tem como característica a extensibilidade. Para isso, foi escolhido e ajustado o software Dynagraph, \cite{dynagraph}.

Para a obtenção das informações espaço-temporais mapeáveis e características do indivíduo, seguem-se três estratégias para resolução do problema: na primeira os dados são analisados como um só grupo (Agrupamento Estático); na segunda os dados são tratados por intervalos pré-definidos; na terceira mapeia-se as evoluções entre intervalos observados.
Dessa forma, esta pesquisa visa indicar os possíveis grupos relacionados espacial e temporalmente.

Por fim, foi desenvolvido um modelo capaz de representar agrupamentos e previsão dinâmicos a partir do software Dynagraph, para avaliação e conclusão dos resultados obtidos em bases dinâmicas.

\section{Organização do Texto}
Esta dissertação está organizada em 5 capítulos. O capítulo 1 apresenta uma
introdução à necessidade da representação e manipulação do agrupamento espaço-temporal
dinâmico, assim como a previsão da formação de novos grupos dinâmicos. Em seguida são apresentados os objetivos,
as hipóteses, a metodologia utilizada e as contribuições. O capítulo 2 constitui a revisão
bibliográfica sobre modelagem com grafos dinâmicos, métodos de agrupamento por densidade, redes dinâmicas,
o Dynagraph, um editor de características e um conjunto de trabalhos relacionados a esta área de conhecimento.
O capítulo 3 apresenta o modelo de agrupamento e previsão em redes dinâmicas. O capítulo 4 destaca
os resultados e comparação dos algoritmos apresentados. O capítulo 5 apresenta as considerações finais
e propostas de trabalhos futuros.








