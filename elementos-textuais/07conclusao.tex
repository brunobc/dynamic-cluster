\chapter{Conclusões e Trabalhos Futuros}
\label{chap:conclusoes-e-trabalhos-futuros}

\section{Considerações finais}
\label{sec:contribuicoes-do-trabalho}

Criação de um método para resolver o Problema de Agrupamento em Grafos Dinâ-
micos e previsão de evolução destes agrupamentos.
A expectativa é de que ao final desta pesquisa, tenha-se uma extensão do Dynagraph
para visualização de agrupamentos dinâmicos e previsão dinâmica.
Utilizar a ferramenta para outros tipos de doenças como: Chikungunya e Zika Vírus.
Finalmente, espera-se que o produto final e os resultados obtidos possibilitem a previsão e
prevenção de novos casos de dengue para um combate efetivo à doença.

\section{Limitações}
\label{sec:limitacoes}

Dentre as dificuldades que podem interferir na execução deste projeto de pesquisa,
as seguintes podem ser citadas:
1. A escassez de estudos relacionando os assuntos abordados: agrupamento, previsão em
dados dinâmicos espaço-temporais, grafos dinâmicos e sistemas web de forma integrada;
2. Obtenção das informações dos focos e casos de dengue geolocalizadas e tempos das
ocorrências. A extração dos dados semanais requer um processo manual em \cite{simda},
pois é necessário o usuário selecionar o ano e a semana correspondente;
3. Visualização dos agrupamentos dinâmicos.
Para contornar as dificuldades apresentadas pretende-se:
1. Automatizar a forma de obtenção dos dados;
2. Utilizar o software Dynagraph como ambiente de suporte à validação e interação com os
resultados dos métodos de agrupamento espaço-temporal utilizados.

\section{Trabalhos Futuros}
\label{sec:trabalhos-futuros}

Em estudos futuros, pretende-se executar o algoritmo em paralelo, a fim de melhorar o desempenho, pois grandes bancos de dados necessitam de alto poder computacional. Além disso, heurísticas mais performáticas podem ser encontradas para determinar os parâmetros de entrada Eps e MinPts.



