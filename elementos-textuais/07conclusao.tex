\chapter{Conclusões e Trabalhos Futuros}
\label{chap:conclusoes-e-trabalhos-futuros}

\section{Conclusões}
\label{sec:conclusoes}

Nesta dissertação foi apresentada uma metodologia para resolver o Problema de Agrupamento em Grafos Dinâmicos e previsão de evolução destes agrupamentos no espaço-tempo aplicado na expansão da dengue e chikungunya.

Foi realizada a integração de um editor de características ao Dynagraph, tendo em vista a facilidade do software em extender para outras aplicações. Com o editor é possível configurar os atributos dos vértices e ligações entre os vértices.

A extração e armazenamento dos dados foi automatizada seguindo a estrutura de dados do Dynagraph. Após obter os dados no formato utilizado em \acrshort{SIMDA} o material foi convertido para a modelagem espaço-temporal no Dynagraph.

No \acrfull{SAD} foi formalizada a metodologia do processo de aquisição de dados, representação, parametrização, execução e análise do processo de previsão espaço-temporal. Foi desenvolvido um \textit{\acrfull{DW}} integrando-se de forma semi-automática ao \acrshort{SIMDA}, acoplado a uma extensão do \emph{Dynagraph} que serve como ambiente de visualização de agrupamentos dinâmicos e previsão dinâmica utilizando os algoritmos de agrupamento dinâmico espaço-temporal: \acrshort{ST-DBSCAN} e \acrshort{ST-IGN} aqui reportados. 

Utilizando como base as análises e os experimentos realizados neste trabalho, conclui-se que o \acrshort{ST-DBSCAN} apresenta melhores resultados do que os obtidos pelo \acrshort{ST-IGN} em relação ao percentual de acerto de casos dentro de algum grupo de previsão. Já o \acrshort{ST-IGN}, apesar de gerar menos grupos de previsão que o \acrshort{ST-DBSCAN}, é melhor em relação a gerar mais grupos de previsão com pelo menos um caso dentro. Destaca-se a eficiência do \acrshort{ST-IGN} em otimizar recursos, por exemplo, direcionando agentes sanitaristas à locais onde será mais provável ocorrer a expansão da endemia.
Sugere-se utilizar os dois algoritmos em conjunto para obtenção de resultados mais precisos, e com isso podendo destacar os pontos de interseção como mais críticos e conduzir os recursos de forma mais otimizada.

A média móvel das variáveis \emph{posição de centros de grupos} e \emph{raio}  foi utilizada com relativo sucesso (momentos críticos de expansão dos casos humanos de dengue e chikungunya) para realizar a previsão dos locais onde estariam os casos humanos na próxima semana epidemiológica.   

Os resultados apresentados demonstram que a aplicação adequa-se a mecanismos de combate a endemias tradicionais para que ações sejam antecipadas, principalmente nos momentos mais críticos de expansão das doenças investigadas. 

\section{Limitações da Metodologia}
\label{sec:limitacoes}

A primeira dificuldade encontrada foi a escassez de estudos (portais de periódicos da \acrshort{CAPES}, \acrshort{IEEE}, Scielo) relacionados aos assuntos abordados: agrupamento, previsão em dados dinâmicos espaço-temporais, grafos dinâmicos e sistemas web de forma integrada. Além disso, obter as informações dos casos de endemias geolocalizadas e tempos das ocorrências. A extração dos dados semanais requer um processo manual em \acrshort{SIMDA}, pois é necessário o usuário selecionar o ano e a semana correspondente.

Para contornar as dificuldades apresentadas na extração dos dados, o processo foi parcialmente automatizado. Outro ponto importante foi utilizar o software \emph{Dynagraph} para visualização dos agrupamentos dinâmico e como ambiente de suporte à validação e interação com os resultados dos métodos de agrupamento espaço-temporal utilizados. Não se pode esquecer das facilidades disponibilizadas na ferramenta Google Maps$^{TM}$ no tratamento dos grupos.

Durante as semanas de baixa incidência de casos de dengue ou chikungunya (menor que 100 casos) a acertividade dos métodos foi bastante comprometida, necessitando para isto de uma verificação mais cuidadosa dos parâmetros mais adequados para estes períodos. Porém de fato era de se esperar que os agrupamentos de casos não pudessem se formar uma vez que a distância entre eles nestes momentos é muito grande.

\section{Trabalhos Futuros}
\label{sec:trabalhos-futuros}

No caso das doenças estudadas, pode-se considerar importante que sejam usados mecanismos de prevenção baseados também na presença dos vetores da doença. Deste modo aumentando a chance de melhor combate às doenças como a dengue e a chikungunya aqui estudadas. 

Estudos de parametrização dinâmica para aumentar a acertividade dos métodos espaço-temporais são necessários. Deste modo será possível obter taxas de acerto elevadas nos momentos de prevenção. O estudo estatístico de séries temporais, baseando-se em momentos passados de períodos semelhantes é promissor.

Para o editor de características espera-se estender as mudanças dos atributos aos grupos de previsão espaço-temporal e permitir uma melhor análise de acordo com as necessidades do tomador de decisão, por exemplo, destacar os grupos de previsão que estão em interseção.

Em estudos futuros, pretende-se executar o algoritmo em paralelo, a fim de melhorar o desempenho, pois à medida que a quantidade de informação em um sistema cresce, ela precisa ser armazenada em algum lugar e de maneira inteligente no espaço e tempo. Além disso, heurísticas com melhor desempenho podem ser encontradas para determinar os parâmetros de entrada ${Eps}$ e ${MinPts}$ do ST-DBSCAN e ST-IGN.

Para o ST-IGN, espera-se encontrar melhores Critérios de Corte da AGM, por exemplo avaliando o conjunto de dados e aplicando algum algoritmo a fim de obter agrupamentos preliminares.

Explorar outros modelos de previsão dinâmica seria válido, a fim de obter melhores resultados dos grupos de previsão, como Redes Neurais Artificiais e Séries Temporais.

Finalmente, espera-se utilizar a ferramenta para outros tipos de doenças epidemiológicas como a Leptospirose, Calazar, Tuberculose e outras; e que o produto final e os resultados obtidos possibilitem a previsão e prevenção de novos casos de endemias para um combate efetivo a essas doenças.


%	ok	Inclui símbolo do IFCE na capa;
%	ok	Falar sobre IFCE na capa;
%	ok	Falar sobre IFCE na contra-capa;
%	ok	Falar sobre IFCE na folha de assinaturas;
%	ok	Melhorar a Introdução;
%	ok	Alterar título do cap. 2;
%	ok	Legenda embaixo das figuras (abnt determina em cima);
%	ok	Evitar frases como “Grandes quantidades” (p.13);
%	ok	Analisar nome do capítulo 2 Conceitos…;
%	ok	Em processos… desses grupo (p.19);
%	ok	Incluir significado de CURE (p.25);
%	ok	Substituir framework por Estrutura computacional, colocar em itálico (p.50);
%	ok	Itálico em St-Outlier (p.52);
%	ok	Referenciar SIMDA e WebDengue (p.72);
%	ok	Traduzir Data Warehouse (p.73);
%	ok	Verificar em referências se coloca data de acesso igual a data da referência;
%	ok	Merge ou mesclar (p.28);
%	ok	Definir I/O (p.27) e outliers;
%	ok	Incluir significado de UTM (p.56);
%	ok	Colocar fig 25 dyn.json em uma única página (p.56);
%   ok  Alterar localização da imagem do WebDengue
%   ok  Definir outliers na introdução (pegar da seção outlier espaço-temporal)
%	ok	Outliers é mencionado antes de ser definido (p.23);
%	ok	Refazer a frase inicial em algoritmo de birch (p.22);
%	ok	Analisar algoritmo de birch 1.a.;
%   ok	Acrescentar hipótese sobre “acertividade da metodologia no tocante do monitoramento da evolução da doença”;
%   ok  Correções em Conceitos e Revisão bibliográfica

% citeautoronline
%	ok	Retirar parênteses de referência (p.25);
%	ok	Remover parênteses de referência (p.22) Birch;
%	ok	Remover parênteses de referência (p.32);
%	ok	Remover parênteses de referência (p.35);
%	ok	Remover parênteses de referência (p.40);
%	ok	Remover parênteses de referência (p.54) 2x;

%	ok	Incluir lista de siglas;
%   ok  Referenciar siglas;

%	ok	Trabalhos relacionados (p.65) verificar item como sendo trabalho relacionado;
%	ok	Analisar Níveis de acertividade, constantes no resumo? (p.82) - conclusão;

%   ok  Na contextualizacao reporte sobre as formas de trabalho espaço-temporal. (Kisilevich2009) 

%	ok	Incluir/apresentar cubo analítico no trabalho escrito;
%	ok	Analisar/incluir gráfico com resultado de 98%;
%   ok  Acrescentar imagem genérica da metodologia (atualmente só tem do SAD)

%   ok  DISCUSSÃO: Dissertacao_AntonioCavalcanteAraujoNeto_Final

%   ok  Ler novos .pdfs
%   ok  Ler tese Detection of Spatial and Spatio-Temporal Clusters (Daniel B. Neill)
%   ok  Ler sobre CDC e TDR https://www.who.int/tdr/en/
%   ok  Ler DENGUE Integracao_de_sistemas_computacionais_e_modelos_lo

%   ok  MELHORAR Trabalhos Relacionados principalmente Estudos Relacionados
%   ok  MELHORAR conclusão

%   ok  analisar se hipóteses correspondem a conclusão:
%   ok  É viabilizado o armazenamento dos dados por uma ferramenta de extração e armazenamento de dados automatizada seguindo a estrutura de dados do Dynagraph.
%   ok  É exequível a integração de um editor de características ao Dynagraph, que é um software extensível.	
%   ok  É realizável a utilização do modelo proposto de agrupamentos em grafos dinâmicos em um ambiente Web.	
%   ok  É possível a criação de um algoritmo capaz de sugerir agrupamentos naturais dinâmicos espaço-temporais de casos de doenças como dengue e chikungunya geolocalizados no tempo.
%   ok  É possível a avaliação da acertividade da metodologia no tocante do monitoramento da evolução da doença.

% ok    Incluir na discussão o motivo de utilizar os valores/parâmetros nos testes - 1km, 500m...

% Foi observado que o...
% Com base nos resultados apresentados, foi possível verificar que...
% Como a técnica proposta segue a abordagem...foi necessário verificar...

% outliers
%   -   MELHORAR assunto sobre outliers
%   -   Ler 10.1.1.352.3123_Survey_OutlierDetectionTemporalData.pdf
%   -   Ler guptaOutlierDetectioninTemporalData_Curso.pdf

%	-	MELHORAR explicação da metodologia (imagem)

%	-	MELHORAR didática na apresentação
%\cite{tao2010}

